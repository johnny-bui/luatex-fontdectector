\documentclass[12pt,a4paper]{article}
\usepackage{luacode,luaotfload}
\usepackage{fontspec}


\begin{document}
\begin{luacode*}

local max = 10000

dofile("makeline.lua")
dofile("commonuse.lua")


tabu_font = {
"MnSymbol12", "MnSymbol10", "MnSymbol9","MnSymbol8","MnSymbol7","MnSymbol6", "MnSymbol5",
"MnSymbol-Bold5","MnSymbol-Bold6","MnSymbol-Bold7", "MnSymbol-Bold8","MnSymbol-Bold9","MnSymbol-Bold10", "MnSymbol-Bold12",
"blsy",
"SkakNew-Diagram","SkakNew-Figurine","SkakNew-Figurine","SkakNew-DiagramT",
"ocrb8","ocrb9","ocrb10","ocrb6","ocrb7","ocrb5",
"cmex10",
"cmmi10",
"cmr10",
"cmsy10",
"esint10",
"eufm10",
"padauk","padaukbook",
"rblmi",
--"Bitter-Italic","Bitter-Bold",
--"bitter",
--"STIXGeneral-Bold","STIXGeneral-Italic","STIXGeneral-Regular","STIXGeneral-BoldItalic",
--"SteveHand",
--"TeXGyreHeros-BoldItalic","TeXGyreCursor-Regular","TeXGyreCursor-Italic",
"CCIcons",
"GFSArtemis",
"garuda",
"kinnari",
"loma",
"msam10",
"msbm10",
"rsfs10",
"norasi",
"sawasdee",
"oxygensansbold",
--"phaistos",
--"purisa",
--"umpush",
"waree",
"wasy10",
"ccicons"
}


function foreachinorder(t, cmp)
	--- # return 2 Lists
	--- # * The first list ist the list of all font family name in sytem and tex, alphabetical sorted
	--  # st. like ['ariel','hveltica']
	--  # 
	--  # * The second one ist a map of all font faces in the font family, st. like
	--  # ['arial'=['arial-itatlic','arial-regular','arial-bold'], 'heveltica'=['heveltica-itatlic','heveltica-regular'] ]
    -- first extract a list of the keys from t
    local font_list = {}
    local font_family = {}
    for k,v in ipairs(t) do
		local font_name = v
		if font_family[1] == nil then
			print ("init font_family")
			font_family[1] = v.familyname
			font_list[v.familyname] = {font_name}
		end
		local last_family_name = v.familyname
		-- print("name ", font_name , " family ", last_family_name)
		if (contains(font_family, v.familyname)) then
			-- print("", font_name , " to family ", last_family_name)
			local font_tab = font_list[last_family_name]
			if not contains(font_tab, font_name) then
				font_tab[#font_tab +1] = font_name
			end
		else
			-- print ("insert new font family ", v.familyname)
			font_family[#font_family+1] = v.familyname
			-- print ("add ", font_name , "to family", v.familyname)
			font_list[v.familyname] = {font_name}
		end
    end
 	-- then sort them   
    table.sort(font_family,cmp)
    return font_family, font_list
end


--- Return the section-entry of the font family
function tex_font_family(font_family)
	print(font_family)
	local section = string.format([[
\section{%s}
]], font_family)
	return section
end

function texFontFace(fontObject)
	local fontLoadreference = fontObject.plainname
	local fontDisplayName = escapeTexChar( fontObject.plainname )
	
	local fontUseCmd,  declareFont = createDeclareFont(fontObject)

	local demoFont =
string.format( [[
\subsection{%s \hspace{3ex} \hfill{\%s ABC abc 123} }
{\footnotesize\begin{verbatim}
%s\end{verbatim}}
\%s ]]
, fontDisplayName, fontUseCmd, 
declareFont,
fontUseCmd
)

	
	return declareFont, demoFont
end

-- myfonts=dofile(fonts.names.path.path)
myfonts=dofile(fonts.names.path.index.luc)

--local out_dir_name = "./texfont/%s.tex"
local out_dir_name = "./texfont/"
--temp_fn = string.format(out_dir_name,"temp")
temp_fn = out_dir_name .. "temp.tex"
file = io.open(temp_fn, "w")
file:write(string.format(doc_prefix,"Font-List"))

--tempfontDeclare = string.format(out_dir_name, "fontdec")
tempfontDeclare = out_dir_name .. "fontdec.tex"
fontDecFile = io.open(tempfontDeclare, "w")

font_family, font_list = foreachinorder(myfonts.mappings, str_sort)

local i = 1
for _ , k in ipairs(font_family) do
	print ('k=' .. k)
	if not contains(tabu_font, k) then
		--- Put an section in output file
		local font_face= font_list[k]
		table.sort(font_face, compareFontObject)
		local fontFamilyName = (font_face[1]["plainname"]):gsub("-?Regular",""):gsub("-?regular","")
		if not contains(tabu_font, font_face[1].fontname) then
			file:write( tex_font_family( escapeTexChar(fontFamilyName) ))
		end
		-- print ('fontface= ' .. dir(font_face, 0))
		-- for eache font face of the font family, put an subsection with sample text
		-- in output dir
		for _, font in ipairs(font_face) do
			local fontObject = font
			if not contains(tabu_font, fontObject.fontname) then
					i = i + 1
					decFontFace, demoFontFace = texFontFace(fontObject)
					fontDecFile:write( decFontFace )
					file:write ('{{')
					file:write( demoFontFace )
					file:write(make_probe_text())
					file:write ('}}\n')
				if i > max then 
					print ("Font face couter reaches max " .. (i-1) )
					break
				end
			end
		end

		if i > max then 
			print ("Font face couter reaches max " .. (i-1) )
			break 
		end
	end
end

file:write(doc_suffix)
file:close()

--lua_cmd = string.format("lualatex -interaction nonstopmode -output-directory=texfont %s", temp_fn)
--print(lua_cmd)
--r = os.execute(lua_cmd)
--print("return value", r)
--if (r == 0) then
	--r = os.execute(lua_cmd)
	--r = os.execute(lua_cmd)
--else
--	print ("call lualatex returned not null valu not null valuee:",r)
--end

fontCollectDirName = "./build/"
for i,k in ipairs(font_family) do
	local font_face=font_list[k]
	if not contains(tabu_font,font_face) then
		local file_name = escapeTexChar(k) .. '.tex'
--		local tex_file = io.open( string.format(out_dir_name,file_name),"w" )
		local tex_file = io.open( fontCollectDirName .. file_name,"w" )
		write_glyphen_test_file(k, font_face, tex_file)
		if  i >= max then 
			print ("Font face couter reaches max " .. (i-1) )
			break 
		end
	end
end

--compile_all_tex = [[for f in `ls texfont/*.tex` ; do lualatex -output-directory=texfont -interaction nonstopmode $f &  done]]
--local r = os.execute(compile_all_tex)
--print("return value", r)
--if (r == 0) then
--	r = os.execute(lua_cmd)
--	r = os.execute(lua_cmd)
--else
--	print ("call lualatex returned not null valu not null valuee:",r)
--end

\end{luacode*}
\end{document}



